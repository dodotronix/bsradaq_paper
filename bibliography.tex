\ifboolexpr{bool{jacowbiblatex}}%
{\printbibliography}{

    \begin{thebibliography}{99}
	
        % http://cds.cern.ch/record/782076/files/CERN-2004-003-V1.pdf?version=2
        \bibitem{LHC_report} BR\"UNING, O. \emph{LHC design report Volume I The LHC
            Main Ring}. Geneva: European Organization for Nuclear Research, c2004-. CERN
            (Series), 2004-003. ISBN 92-9083-224-0

        \bibitem{abort_gap_monitor}De Santis, S., Beche, J., Byrd, J., Placidi, M.,
            Turner, W. \& Zolotorev, M. \emph{Development of an Abort Gap Monitor for the
            Large Hadron Collider}.  (2004), \url{https://cds.cern.ch/record/823343}

        % https://cds.cern.ch/record/1272172/files/CERN-BE-2010-026.pdf
        \bibitem{bsra_operation}Bart Pedersen, S., Lefevre, T., Bravin, E., Boccardi, A.,
            Goldblatt, A., Jeff, A., Roncarolo, F. \& Fisher, A. \emph{First Operation of
            the Abort Gap Monitors for LHC}.  (2010,5), \url{http://cds.cern.ch/record/1272172}


    \end{thebibliography}
} % end \ifboolexpr
