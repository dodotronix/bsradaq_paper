% !TeX spellcheck = en_GB
% !TeX program = lualatex

\documentclass[a4paper,
               %boxit,        % check whether paper is inside correct margins
               %titlepage,    % separate title page
               %refpage       % separate references
               %biblatex,     % biblatex is used
               keeplastbox,   % flushend option: not to un-indent last line in References
               %nospread,     % flushend option: do not fill with whitespace to balance columns
               %hyphens,      % allow \url to hyphenate at "-" (hyphens)
               %xetex,        % use XeLaTeX to process the file
               %luatex,       % use LuaLaTeX to process the file
               ]{jacow}
%
% ONLY FOR \footnote in table/tabular
%
\usepackage{pdfpages,multirow,ragged2e} %
%
% CHANGE SEQUENCE OF GRAPHICS EXTENSION TO BE EMBEDDED
% ----------------------------------------------------
% test for XeTeX where the sequence is by default eps-> pdf, jpg, png, pdf, ...
%    and the JACoW template provides JACpic2v3.eps and JACpic2v3.jpg which
%    might generates errors, therefore PNG and JPG first
%
\makeatletter%
	\ifboolexpr{bool{xetex}}
	 {\renewcommand{\Gin@extensions}{.pdf,%
	                    .png,.jpg,.bmp,.pict,.tif,.psd,.mac,.sga,.tga,.gif,%
	                    .eps,.ps,%
	                    }}{}
\makeatother

% CHECK FOR XeTeX/LuaTeX BEFORE DEFINING AN INPUT ENCODING
% --------------------------------------------------------
%   utf8  is default for XeTeX/LuaTeX
%   utf8  in LaTeX only realises a small portion of codes
%
\ifboolexpr{bool{xetex} or bool{luatex}} % test for XeTeX/LuaTeX
 {}                                      % input encoding is utf8 by default
 {\usepackage[utf8]{inputenc}}           % switch to utf8

\usepackage[USenglish]{babel}

\usepackage[table]{xcolor}
%
% if BibLaTeX is used
%
\ifboolexpr{bool{jacowbiblatex}}%
 {%
  \addbibresource{jacow-test.bib}
  \addbibresource{biblatex-examples.bib}
 }{}
\listfiles

\usepackage{pgf}

\graphicspath{
    {pictures/photos},
    {pictures/ipe}}
    
\newcommand{\mathdefault}[1][]{}

%%   Lengths for the spaces in the title
%%   \setlength\titleblockstartskip{..}  %before title, default 3pt
%%   \setlength\titleblockmiddleskip{..} %between title + author, default 1em
%%   \setlength\titleblockendskip{..}    %afterauthor, default 1em
\usepackage{enumitem}
\setlist{noitemsep}

\begin{document}

\title{LHC abort gap monitor electronics upgrade}

\author{P. Pacner\thanks{petr.pacner@cern.ch}, Brno University of Technology, Brno, Czech Republic \\
		S. B. Pedersen, D. Belohrad, M. M. Nieto, S. Mazzoni, CERN, Geneva, Switzerland}
	
\maketitle

\begin{abstract}
    The LHC Abort Gap Monitor (AGM) is part of the LHC machine protection
    system (MPS) and is designed to measure the particle population in a 3us
    wide region known as the "abort gap." This region needs to be kept empty to
    ensure safe beam dumps. The AGM captures the synchrotron light generated in
    the visible part of the spectra and converts it into an electric signal.
    This signal is then processed by an acquisition system and can trigger the
    ‘abort gap cleaning’ process.\\
    The current AGM, which has been in operation since 2010, uses an analogue
    integrator ASIC and a 40 MHz analogue-to-digital (ADC) converter to provide
    the particle population information. However, this solution is now
    considered obsolete and is being replaced by a digital signal processing
    approach. Working directly in the digital domain not only offers more
    scalability but also better determinism and reliability. \\
    This work presents the new technical solution for the acquisition chain,
    compares the characteristics of both implementations, and showcases recent
    measurements conducted on the LHC ion and proton beams.
\end{abstract}

\section{Introduction}
The Large Hadron Collider (LHC) operates with high-energy beams that possess
the capacity to melt metals and induce quenching in magnets, necessitating
precise control measures to manage the beam when exiting vacuum compartments.
The LHC beam consists of 2808 bunch slots, organized in bunch trains with a
3~$\mu$s gap, determined by kicker magnet specifications for controlled
beam expulsion from vacuum chambers.\\
This paper addresses the imperative need for an electronics upgrade in the AGM
system. The current AGM, operational since 2010, relies on an analogue
integrator ASIC and a 40 MHz analogue-to-digital converter (ADC) for particle
population measurements. Recognizing the obsolescence of this analogue
solution, the paper introduces a transition to a digital signal processing
approach. Working directly in the digital domain not only offers more
scalability but also better determinism and reliability.\\
The following sections delve into the current AGM system, its limitations, the
rationale for the upgrade, the new technical solution, a comparative
measurements.

\section{Current system design}
        %description of the system
        %Utilization of Analogue Integrator ASIC
        %40 MHz Analogue-to-Digital Converter (ADC)
        %Operation Since 2010
        %Reasons for Obsolescence
        %Digital Signal Processing Approach why and outcomes
    \subsection{Technical Solution}
        %Utilization of Analogue Integrator ASIC
        %40 MHz Analogue-to-Digital Converter (ADC)
        %Operation Since 2010
        %Reasons for Obsolescence
        %Digital Signal Processing Approach why and outcomes

\section{New fully digitized AGM}
    \subsection{}

\section{Comparison of Implementations}
    %compare the changes and comment the expected changes

\section{Recent Measurements}

\section{Discussion}
    \subsection{Significance of the Upgrade}
    \subsection{Future Implications}
    % \subsubsection{Potential for Further Enhancements or Modifications}
    % \subsubsection{Integration with Other Accelerator Technologies}

\section{Conclusion}
\section{Acknowledgments}

\section{ACKNOWLEDGEMENTS}
Any acknowledgement should be in a separate section directly preceding
the \textbf{REFERENCES} or \textbf{APPENDIX} section.

\section{APPENDIX}
Any appendix should be in a separate section directly preceding
the \textbf{REFERENCES} section. If there is no \textbf{REFERENCES} section,
this should be the last section of the paper.

\ifboolexpr{bool{jacowbiblatex}}%
{\printbibliography}{

    \begin{thebibliography}{99}
	
        % http://cds.cern.ch/record/782076/files/CERN-2004-003-V1.pdf?version=2
        \bibitem{LHC_report} BR\"UNING, O. \emph{LHC design report Volume I The LHC
            Main Ring}. Geneva: European Organization for Nuclear Research, c2004-. CERN
            (Series), 2004-003. ISBN 92-9083-224-0

        \bibitem{abort_gap_monitor}De Santis, S., Beche, J., Byrd, J., Placidi, M.,
            Turner, W. \& Zolotorev, M. \emph{Development of an Abort Gap Monitor for the
            Large Hadron Collider}.  (2004), \url{https://cds.cern.ch/record/823343}

        % https://cds.cern.ch/record/1272172/files/CERN-BE-2010-026.pdf
        \bibitem{bsra_operation}Bart Pedersen, S., Lefevre, T., Bravin, E., Boccardi, A.,
            Goldblatt, A., Jeff, A., Roncarolo, F. \& Fisher, A. \emph{First Operation of
            the Abort Gap Monitors for LHC}.  (2010,5), \url{http://cds.cern.ch/record/1272172}


    \end{thebibliography}
} % end \ifboolexpr


%%%\include{annexes-A4}

\end{document}

